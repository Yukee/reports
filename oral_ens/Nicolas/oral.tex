\documentclass[7pt,a4paper, french]{report}

\usepackage[T1]{fontenc}
\usepackage[utf8]{inputenc} %caractères spéciaux
\usepackage[english, french]{babel} %français
\usepackage[left=2cm, right=2cm, top=2cm, bottom=2cm]{geometry}
\usepackage{graphicx}

\usepackage{amsmath}
\usepackage{amssymb}
\usepackage{mathrsfs} %maths

\usepackage{xspace} %symboles extensifs

\begin{document}

\section*{Entretien}

\textit{On a une heure pour preparer l'entretien (lire le texte, repondre aux questions), puis une heure de passage. Je n'ai lu le texte qu'une fois, et ne m'en suis servi que pour donner l'ordre de grandeur de la liaison carbone-carbone... }
\subsection*{Texte :}
\textit{Le graphene.} Reflets de la physique, Numéro 25, juillet-août 2011.
\subsection*{Questions sur le texte}
\begin{itemize}
\item Definir : isolant, conducteur, semi-conducteur (notions de gap, de bandes).
\item Quelle est la taille typique de la liaison carbone-carbone dans un nanotube ?
\item Techniques de microscopie utilisables pour observer les nanotubes ?
\item Difference entre liaison covalente et liaison de Van der Waal ?
\end{itemize}

\subsection*{Questions generales}
\begin{itemize}
\item Definir diamagnetique, paramagnetique, ferromagnetique.
\item Calculer le rapport entre le moment cinetique et le moment magnetique.
\item Definir les observables moment cinetique et moment de spin en mecanique quantique, donner des exemples pour des particules elementaires.
\end{itemize}

\subsection*{Question experimentale}
Comment determiner l'ellipticite d'une polarisation circulaire ?

\subsubsection*{Remarques :} \textit{Les examinateurs etaient sympa, me signalant les erreurs que je commetais en me posant des questions du genre ``Vous etes sur ?''. Dans l'ensemble ils semblaient attendre des reponses tres basiques aux questions (par ex. dans la question sur le moments cinetiques et de spins, ils semblaient juste vouloir que je dise que $\hat{L} = \hbar j(j+1)$...). Ils voulaient que je donne beaucoup d'ordres de grandeur (par ex. ODG du gap d'un semi conducteur, de la taille d'une bande d'energie).}

\section*{TP}
\textit{Le but du TP etait de construire un microscope sur un banc optique en positionnant un objectif de microscope demonte et une lentille convergente faisant office d'oculaire. On avait ensuite a observer un ``object inconnu'' (en fait une grille) place sur une lame en verre, a l'aide du miscroscope.}
\subsection*{I}
\begin{itemize}
\item Mesurer la focale $f_1^{'}$ de l'oculaire. Precision ? \textit{On ne peut pas mesurer la focale directement, car on ne sait pas ou se trouve le centre optique de l'objectif. En plus la mesure de $\bar{OA}$ serait peu precise car l'objectif ayant une focale tres courte, cette longueur est de l'ordre du mm. L'idee est de passer par le grandissement $G$ avec la formule $\bar{OA'}=f_1^{'} \cdot (1-G)$. On peut tracer la droite$\bar{OA'}$ en fonction de $G$, dont la pente est $f_1^{'}$.}
\item Pourquoi est-il conseille d'utiliser l'objectif avec le grandissement indique par le constructeur ?
\item Mesurer l'ouverture numerique (sinus de l'angle maximal du cone forme par les rayons entrant dans le systeme optique) de l'objectif.
\end{itemize}

\subsection*{II}
\begin{itemize}
\item Faire le schema d'un microscope. A quoi sert l'oculaire ?
\item Montrer que $f_1^{'} = - \bar{F_{1}^{'}F_{2}}/G$.
\item Fabriquer le microscope.
\item Comment mesurer la longueur d'un objet observe ? \textit{On peut tout betement mettre une grille transparente (il y en avait une sur la table) dans le plan de formation de l'image par l'objectif de l'objet observe.}
\item Caracteriser l'objet inconnu en l'observant au microscope.
\item Caracteriser l'objet inconnu en observant sa figure de diffraction avec un laser. Precision ?
\end{itemize}

\textbf{Remarques :} \textit{Examinateurs sympa.}


\end{document}
