

\Chapter{General presentation}{The Lifshitz model}

\section{The Lifshitz action and its main features}

\subsection{The Lifshitz action}

The Lifshitz model aims at describing a number of physical many-body systems. They share a common intriguing feature: having a so called modulated - or stripped - phase. In this phase, the order parameter is spatially periodic in one or several directions of space. The subspace spanned by these direction will from now be labelled $\sslash$. The hyperplan orthogonal to this modulation direction will be labelled $\perp$.

Typically, the phase diagram of such a physical system will resemble this one

(figure to be drawn)

where $\rho_0$ is a tunable parameter whose precise meaning depends on the physical nature of the studied system. We will however take for granted that it is a scalar.

The Lifshitz model is a field theory, describing a vector field $\phi$ whose components will be denoted $\phi_i$. To write the action for the Lifshitz model, we chose a basis $(\mathbf{e_n})_{1 \leq n \leq d}$. We decide that this basis is such that its first $m$ vectors span the $m$ dimensional $\sslash$ subspace, while of course the remaining $d-m$ base vector span the $\perp$ subspace. In this basis, the Lifshitz action is
\begin{equation}
S = \int_x \sum_{i=1}^N \left( \frac{1}{2} \left(\sum_{n_\perp=m+1}^{d}\frac{\partial \phi_i}{\partial x_{n_\perp}} \mathbf{e_{n_\perp}}\right)^2 + \frac{\rho_0}{2} \left(\sum_{n_\sslash=1}^{m}\frac{\partial \phi_i}{\partial x_{n_\sslash}} \mathbf{e_{n_\sslash}}\right)^2 + \frac{\sigma_0}{2} \left(\sum_{n_\sslash=1}^{m} \frac{\partial^2 \phi_i}{\partial x_{n_\sslash}^2} \mathbf{e_{n_\sslash}}\right)^2 \right) + U(\phi)
\end{equation}
where $U$ is an almost completely arbitrary potential. We only ask for him to have the $O(N)$ symmetry, ie to be a function of
\begin{equation}
\rho \define \frac{\phi_i \phi_i}{2}
\end{equation}
From now on we will use the self-explanatory shorthand notation
\begin{equation}
S = \int_x \left( \frac{1}{2}(\p{\perp} \phi)^2 + \frac{\rho_0}{2}(\p{\sslash} \phi)^2 + \frac{\sigma_0}{2} (\p{\sslash}^2 \phi)^2 + U(\rho) \right)
\end{equation}

We see that this action closely ressemble the well known action of the $O(N)$ model
\begin{equation}
S_{O(N)} = \int_x \left( \frac{1}{2}(\partial \phi)^2 + U(\rho) \right)
\end{equation}
Namely, we recover it if we set $\rho_0 = 1$ and $\sigma_0 = 0$. We see that what differentiate the Lifshitz and $O(N)$ action is on one hand the presence a non trivial (\textit{ie} different from 1) $\rho_0$, breaking the $O(N)$ invariance, and on the other hand the presence of an extra term involving a laplacian squared. Clearly, these two modifications must be responsible for the appearance of spatially modulated structures, but why exactly? 
We can gain a useful intuition of why a spatially modulated structure is closely linked to the existence of a laplacian squared term in the action by looking at a lattice version of our model. 

\subsection{A discrete counterpart: the anisotropic Ising model}

Strocto sensu the discrete counterpart of the Lifshitz model would be an anisotropic Heisenberg model, but to simplify things - without changing the essence of the argumentation - we consider an anisotropic Ising model instead.

First, let us consider a chain of Ising spins with the Hamiltonian
\begin{equation}
H_{\text{chain}} = - J \sum_i S_i S_{i+1}
\end{equation}
We know that if $J$ is positive, the interaction is ferromagnetic, whereas is $J$ is negative, the interaction is antiferromegnetic. 
The antiferromagnetic order already shows some kind of spatial modulation, but it only exists at zero temperature! 
The idea to make a spatially modulated order survive at non zero temperatures is to consider a second neighbours \textit{antiferromagnetic} interaction, together with a first neighbours \textit{ferromagnetic} interaction:
\begin{equation}
H_{\text{chain}} = - J_1 \sum_i S_i S_{i+1} - J_2 \sum_i S_i S_{i+2}
\end{equation}
 The competition between ferromagnetic and antiferromagnetic interactions may well produce a spatial modulation of the spins at non zero temperatures, at least for some values of the interaction strenghts ratio $J_2/J_1$. However, for a long range order to exist at finite temperature, we need to work in two dimensions or more, \textit{ie} to trade our spin chain for a spin lattice:
 \begin{equation}
 H_{\text{lattice}} = - \sum_i \left( J_0 \sum_{\delta_\perp} S_i S_{i+\delta_\perp} + J_1 \sum_{\delta_\sslash} S_i S_{i+\delta_\sslash} + J_2 \sum_{\delta_\sslash} S_i S_{i+2 \delta_\sslash} \right)
 \end{equation}
 The existence of a stripped phase is a well known feature of this model \cite{ANNNI}, called the ANNNI (axial next-nearest neighbour Ising) model.
 
 Now, what is the link between this discrete spin lattice hamiltonian, and our continuous action?
First, note that a sum on nearest neighbours can be rewriten in terms of a discrete laplacian on the lattice, while a sum on next-nearest neighbours involves a discrete laplacian squared:
\begin{equation}
H_{\text{lattice}} = -\sum_i \left( \kappa S_i^2 + J_0 S_i \Delta_\perp S_i + (J_1 + 4 J_2) S_i \Delta_\sslash S_i - J_2 S_i \Delta_\sslash^2 S_i \right)
\end{equation}
with
\begin{align}
\Delta_\sslash S_i = \sum_{\delta_\sslash} S_{i-\delta_\sslash} - 2 S_i + S_{i+\delta_\sslash} \\
\Delta_\sslash^2 S_i = \sum_{\delta_\sslash} -S_{i-2\delta_\sslash} +  4 S_{i-\delta_\sslash} - 4 S_i + 4S_{i+\delta_\sslash} - S_{i+2\delta_\sslash}
\end{align}
\section{Physical applications of the model}

