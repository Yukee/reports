\Chapter{The renormalization procedure}{Introduction to the nonperturbative renormalization techniques}

At the so-called Lifshitz critical point, three phases intersect. This is rather unsual, so we expect the physics of the vicinity of this point to be of special interest. To investigate it, we would like to compute the critical exponents associated to this transition point. To this end, we used the powerful machinery of the renormalization group, and more precisely of one particular implementation of the renormalization ideas: the nonpertubative renormalization group.

In this chapter we propose first a very general introduction to the ideas and concepts of the renormalization. Then we focus on the nonperturbative renormalization group techniques.

\section{Introduction to the renormalization group}

\section{The nonperturbative renormalization group}
