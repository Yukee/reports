\documentclass[11pt,a4paper]{report}
\usepackage[utf8]{inputenc}
\usepackage[T1]{fontenc}
\usepackage{amsmath}
\usepackage{amsfonts}
\usepackage{amssymb}
\usepackage{makeidx}
\usepackage{graphicx}
\usepackage[left=2cm,right=2cm,top=2cm,bottom=2cm]{geometry}


%% user-defined commands 
\newcommand\Chapter[2]{
  \chapter[#1: {\itshape#2}]{#1\\[2ex]\Large\itshape#2}
}

\author{Nicolas Macé}
\title{Segregation effects in a granular free-surface flow}
\begin{document}

\Chapter{General presentation}{What is a granular material?}

\section{A polymorphic object}

A granular material is a system composed of a huge number of small interaction units, called \textit{grains}. 
To understand this definition, we need to explain exactly how \textit{huge} we expect the number of grains to be, and exactly how \textit{small} we expect the grains to be.

By "a huge number of grains", we mean a number so huge that we can start using macroscopic quantities to describe the collection of grains. In other words, we want to be able to see the collection of grains as a whole, which we will call \textit{the} granular material.
What is a typical number of grains in a granular material? TO answer this question, it is perhaps useful to look at some examples.
A collective behaviour can be observed, either by direct observations or by finite-elements simulations CIT in a pile of sand of only DAT grains.
We know from our daily morning experiments that we can consider the sugar we pour in our tea or our coffee as a granular flow, whereas it is only constituted of DAT grains.
During this report, we will talk a lot about an experiment implying a granular flow of roughly DAT huge grains.
We see that this "huge number" can be surprisingly small compared for example to the number of molecules in one cubic meter of air ($10^{25}$ molecules at usual temperature and pressure), or even to the number of water molecules in a water droplet ($10^{15}$ in a $500$ $\mu m$ diameter droplet).

Now, what do we mean by "small grains"?
In fact, this number can be surprisingly large!
Examples: cereal silos DAT (mm). Pyroclastic flows DAT (cm). Debris avalanche DAT (m). 
In our case: DAT (cm).
At such scales, gravity plays an important role and in fact, gravitational and contact forces are of comparable intensity. This situation is totally impossible in a gas or in a liquid. A water molecule for example, is so small that its individual behaviour in dictated mainly by Van der Walls forces.
The gravitational forces on a grain may force it to stop, or prevent it from moving (accordingly to Coulomb's friction law).
This is a situation we all experienced when pouring down some sugar. At first, the sugar refuses to flow. This is because the existence of important frictions between grains, driven by gravity. 
In those situations, the granular material can be seen as a solid body.
When flowing, the material behaves in a liquid-like fashion. But we can already note some essential differences with a flowing  liquid: a granular flow can go down a slope slipping while the layer of liquid the closest to the base of the inclined plan is not slipping. This is once again a consequence of the no-so-small size of the grains. Those lying against the base are slipping because they are pulled down by gravity. 
On the other hand, gravitational forces on water molecules of the external layer are neglectables compared to Van der Walls forces. 
Finally, a granular material can behave as a gas in some very special situations. It is the case of the sand in a sand storm, or of the snow particles forming the upper, dilute layer of a snow avalanche. In these cases forces exerted by the flow of air on the grains are so intense that gravitational effects are neglectables. 
Those granular materials have the properties of a gas. One can for example define the "temperature" of such a material CIT.

\section{Granular flows}

"In the top of an hourglass, sand is this strange solid. It's at the verge of being a solid, it flows through the middle as something like a liquid, and then it is a solid again." Jaeger

As exemplified by Jaeger, granular materials such as sand often behave in a liquid like fashion : they flow, they take the form of their container and they have an horizontal surface. However they are far from being liquids. For example sand falling at the bottom of the hourglass forms a pile, not a flat liquid-like surface.
This is because a granular material is highly dissipative. When receiving a energy it tends to disperse it rather than conserve it. This is why it takes a significant amount of energy to make a granular material like a liquid. Moreover as this energy is constantly dispersed inside the material, you have to bring energy to the material continuously in order to keep it flow. 
At low speed, grains tend to assemble and causing the flow to stop suddenly. This can be seen as a phase transition called the jamming transition.
These two effects, jamming and dispersion, are consequences of the relatively large size of the grains forming a granular material. Smaller grains would be able to reach thermal equilibrium thus behaving like a liquid or a gas.

In fact, a granular material is by definition composed of grains sufficiently large not to be able to reach thermal equilibrium, or to be subject to brownian motion\footnote{Particles sufficiently small to be subject to brownian motion, but still very large compared to molecules forming a liquid or a gas are called colloids.}.
This is why smallest grains in a granular material have a radius of the order of the micrometer.
There is no upper bound for the grains' radius. Ice floes forming the polar ocean's ice cap ranging from a few metres to several kilometres can be seen as the giant grains of a granular material.

The most spectacular examples of granular flow are probably debris and pyroclastic flows.
Debris flows usually occur in mountains, after a forest fire followed by intense rainfalls. The soil is then covered in ashes turning into mud because of the addition of rainwater. The mud begin sliding, carrying with it logs, branches, boulders and even entire trees. All these objects can be seen as grains, though they are of very different shape and size. We will see later that the existence of different sizes of grains in a granular material lead to a spectacular effect called segregation.
\end{document}