\documentclass[11pt,a4paper]{report}
\usepackage[utf8]{inputenc}
\usepackage[T1]{fontenc}
\usepackage{amsmath}
\usepackage{amsfonts}
%\usepackage{amssymb}
\usepackage{makeidx}
\usepackage{graphicx}
\usepackage[left=2cm,right=2cm,top=2cm,bottom=2cm]{geometry}


%% user-defined commands 
\newcommand\Chapter[2]{
  \chapter[#1: {\itshape#2}]{#1\\[2ex]\Large\itshape#2}
}
\newcommand{\p}[2]{\ensuremath{\frac{\partial {#1}}{\partial {#2}}}}
\newcommand{\tot}[2]{\ensuremath{\frac{d {#1}}{d {#2}}}}

\newcommand{\vs}{\ensuremath{\mathbf{v}^s}}

\author{Nicolas Macé}
\title{Segregation effects in a granular free-surface flow}
\begin{document}

\Chapter{General presentation}{What is a granular material?}

\section{A polymorphic object}

A granular material is a system composed of a huge number of small interaction units, called \textit{grains}. 
To understand this definition, we need to explain exactly how \textit{huge} we expect the number of grains to be, and exactly how \textit{small} we expect the grains to be.

By "a huge number of grains", we mean a number so huge that we can start using macroscopic quantities to describe the collection of grains. In other words, we want to be able to see the collection of grains as a whole, which we will call \textit{the} granular material.
What is a typical number of grains in a granular material? TO answer this question, it is perhaps useful to look at some examples.
A collective behaviour can be observed, either by direct observations or by finite-elements simulations CIT in a pile of sand of only DAT grains.
We know from our daily morning experiments that we can consider the sugar we pour in our tea or our coffee as a granular flow, whereas it is only constituted of DAT grains.
During this report, we will talk a lot about an experiment implying a granular flow of roughly DAT huge grains.
We see that this "huge number" can be surprisingly small compared for example to the number of molecules in one cubic meter of air ($10^{25}$ molecules at usual temperature and pressure), or even to the number of water molecules in a water droplet ($10^{15}$ in a $500$ $\mu m$ diameter droplet).

Now, what do we mean by "small grains"?
In fact, this number can be surprisingly large!
Examples: cereal silos DAT (mm). Pyroclastic flows DAT (cm). Debris avalanche DAT (m). 
In our case: DAT (cm).
At such scales, gravity plays an important role and in fact, gravitational and contact forces are of comparable intensity. This situation is totally impossible in a gas or in a liquid. A water molecule for example, is so small that its individual behaviour in dictated mainly by Van der Walls forces.
The gravitational forces on a grain may force it to stop, or prevent it from moving (accordingly to Coulomb's friction law).
This is a situation we all experienced when pouring down some sugar. At first, the sugar refuses to flow. This is because the existence of important frictions between grains, driven by gravity. 
In those situations, the granular material can be seen as a solid body.
When flowing, the material behaves in a liquid-like fashion. But we can already note some essential differences with a flowing  liquid: a granular flow can go down a slope slipping while the layer of liquid the closest to the base of the inclined plan is not slipping. This is once again a consequence of the no-so-small size of the grains. Those lying against the base are slipping because they are pulled down by gravity. 
On the other hand, gravitational forces on water molecules of the external layer are neglectables compared to Van der Walls forces. 
Finally, a granular material can behave as a gas in some very special situations. It is the case of the sand in a sand storm, or of the snow particles forming the upper, dilute layer of a snow avalanche. In these cases forces exerted by the flow of air on the grains are so intense that gravitational effects are neglectables. 
Those granular materials have the properties of a gas. One can for example define the "temperature" of such a material CIT.

\section{Granular flows}

"In the top of an hourglass, sand is this strange solid. It's at the verge of being a solid, it flows through the middle as something like a liquid, and then it is a solid again." Jaeger

As exemplified by Jaeger, granular materials such as sand often behave in a liquid like fashion : they flow, they take the form of their container and they have an horizontal surface. However they are far from being liquids. For example sand falling at the bottom of the hourglass forms a pile, not a flat liquid-like surface.
This is because a granular material is highly dissipative. When receiving a energy it tends to disperse it rather than conserve it. This is why it takes a significant amount of energy to make a granular material like a liquid. Moreover as this energy is constantly dispersed inside the material, you have to bring energy to the material continuously in order to keep it flow. 
At low speed, grains tend to assemble and causing the flow to stop suddenly. This can be seen as a phase transition called the jamming transition.
These two effects, jamming and dispersion, are consequences of the relatively large size of the grains forming a granular material. Smaller grains would be able to reach thermal equilibrium thus behaving like a liquid or a gas.

In fact, a granular material is by definition composed of grains sufficiently large not to be able to reach thermal equilibrium, or to be subject to brownian motion. Grains are by definition interacting only by friction and collision\footnote{Smaller particles like powders or colloids are subject to other forces, namely Van der Walls forces or brownian motion forces.}.
This is why the smallest grains have a typical radius of $100$ micrometer.
There is no upper bound for the grains' radius. Ice floes forming the polar ocean's ice cap ranging from a few metres to several kilometres can be seen as the giant grains of a granular material.

The most spectacular examples of granular flow are probably debris and pyroclastic flows.
Debris flows usually occur in mountains, after a forest fire followed by intense rainfalls. The soil is then covered in ashes turning into mud because of the addition of rainwater. The mud begin sliding, carrying with it logs, branches, boulders and even entire trees. All these objects can be seen as grains, though they are of very different shape and size. We will see later that the existence of different sizes of grains in a granular material lead to a spectacular effect called segregation.
Pyroclastic flows occur during or shortly after a volcanic explosive eruption. The mixture of rocks and hot gases ejected by the volcano goes down its flanks at high speed (up to 700 $km \cdot s^-1$).

It is not surprising that many examples of granular flows are fund in the industry. After all, sand is second most used material by human beings after water. 
Apart from sand, one can observe granular flows in cereal or powder silos, or in various pharmaceutical production lines.

\section{Modelling a granular flow}

Be cause a granular is a highly dissipative system, it will often not be able to explore all of its accessible phase space. Instead of reaching its equilibrium state, it will dwell into a metastable state.
This make it difficult to describe it using the tools of statistical mechanics. For example, doing an ensemble average over all accessible configurations of the system is not possible.
Since a granular flow resemble a liquid flow, it will be appropriate to use the tools a fluid mechanics to describe it. This it what we will do in this report.

More precisely, we will make use of the Navier-Stokes equation from fluid mechanics. To derive this equation, we have to write the momentum and force balance on a volume of grains sufficiently small to be considered infinitesimal, but sufficiently large for quantities such as pressure or velocity o be averaged over this volume. A volume verifying these properties is called a control volume. 
A sand flow is typically high of $20$ or $30$ sand grains, so it is questionable that a control volume can be defined for such a system. In such cases,  however, results from computations using a fluid-like description are in good qualitative and quantitative agreement with experiments (cf for example \cite{midi}).

A granular flow is a large collection of microscopic interacting grains. However it does not necessarily mean that it makes it hard to describe description at the macroscopic level. 
Liquid and gases, also constituted of many interacting bodies, are well described using the tools of statistical mechanics.
Since a granular material is a highly dissipative system, it will not be able to explore all of the available phase space. Therefore it will often not reach its equilibrium state and instead dwell into a \textit{metastable state}, as exemplified on fig FIG: a sand pile is formed; the equilibrium state would have the lowest possible centre of mass.
At the microscopic level, because of the inelasticity of shocks and the dissipation of friction forces, grains are not able to move randomly. 
The result is that doing ensemble average over all possible configurations of the system is not relevant, which makes the application of statistical mechanics to granular materials particularly challenging.

Since granular flows behaves like a fluid, an idea is to apply the concepts of fluid mechanics to these systems. That is the approach I used during my internship.
More specifically, we make use of the Navier-Stokes equation describing the temporal evolution of a velocity field - for us, the velocity of the flowing grains. In fluid mechanics, the evolution of the velocity field is driven by the evolution of the pressure and force fields. It is the same for granular materials. Forces exerted on the grains are the gravity, and friction forces. Friction forces are of 2 types: the friction exerted on the grains by the bed the granular material is slipping on, and the friction exerted on the grains by the grains themselves. The latter is modelled by a viscosity term. See APPENDIX for a derivation of the Navier-Stokes equation applied to granular materials.

Contrary to fluids, granular materials have a \textit{stop angle} (cf fig FIG). 
This is modelled by a \textit{velocity-dependant} viscosity. This classifies granular flows as non-newtonian fluids. For granular flows, there is no general, universally valid theory. The velocity dependency of the viscosity term is devised on experimentally. There is good hope, however, that we can use a reasonably universal\footnote{ie independent from the experimental setup} form for the viscosity. See for example \cite{pouli}.
A non-newtonian fluid viscous term is non-linear. This non linearity makes it particularly difficult to study granular flows, and can also lead to ill-posedness problems. I talk in more details about ill-posedness in APPENDIX.

Another special feature of granular flows is segregation. Most of this report will focus on it. The following section is a general presentation of the segregation effect.

\section{Segregation}

\subsection{The effect}

Segregation occurs in granular materials composed of grains of different sizes. What we call segregation is the counter-intuitive effect making the largest grain go at the top of the flow, and the small grains sink at the bottom of the flow. 
It is sometimes called the "Brazil-nut effect" because in a mixture of nuts, the Brazil-nuts being the largest "grains", then tend to go at the surface. I personally prefers the appellation "muesli effect" since it is at play in our morning mueslis! We always find the largest "muesli particles" such as nuts or banana slices at the surface of our cereal bowl, and the smallest ones at the bottom. 

We can explain qualitatively this effect in several ways. A natural way of seeing the segregation effect is to consider that the small particles will fall in the gaps opening between the large ones. Gaps open between large particles when some energy is brought to the system, for example through shaking, or mixing with a spoon if it is a muesli mixture, or through sheering if it is a granular flow.

But this explanation assume that large particles are in a sufficient number to form a "matrix" through the holes of which the small grains can fall. 
In a mixture of earth (small grains) and rocks (large grains), this assumption is no longer valid. However anyone who has tried labourer a field knows that it makes rocks go to the surface. Here again, labourer brings some energy to the system, and it is sufficient to start segregation. 
In that case, the segregation effect can be explained considering that small grains will seize every opportunity to slip between large ones, and so, by force imbalance, large grains will go upward. Since this is an irreversible process, it will progressively make large grains go to the surface.

The net result of segregation is the \textit{advection} of small grains downwards, and the advection of large grains upwards. We will model this effect using an advection equation.

\subsection{The advection equation}

We are interested in modelling a the segregation effect using the fewest possible hypothesis on our physical system. We will discuss in more details the setup of the system I actually studied during my internship, in the next chapter. 
For now, let us just assume that we have a granular flow, constituted of a mixture of 2 different grain types: small and large ones. This assumption is commonly made, since it simplifies greatly the equations, but still captures the essential physical features of segregation. For a more general theory, see \cite{3phase}.

The quantity we are interested in modelling is the ratio of small grains $\phi^s(r)$ and the ratio of large grains $\phi^l(r)$ in the neighbourhood of $r$. More precisely, if $d\tau$ is a small volume of granular material around $r$, and if $d\tau^s$ is the fraction of this volume occupied by small particles, and $d\tau^l$ the fraction occupied by large ones, we have
\begin{align}
	\phi^s = \frac{d\tau^s}{d\tau^l + d\tau^s} = \frac{d\tau^s}{d\tau }\\
	\phi^l = \frac{d\tau^l}{d\tau}
\end{align}
We can focus on $\phi^s \equiv \phi$ and completely forget about $\phi^l$ since $\phi^l = 1 - \phi$. $\phi$ is a non-dimensional number but we will call it \textit{concentration in small particles} par abus de language.
Since the number of particles is a conserved quantity, so is the concentration $\phi$. Since the concentration of small particles is conserved on any volume $\mathcal{V}$ we have

\begin{equation}
	\tot{}{t} \int_\mathcal{V} \phi(r) dr = 0
\end{equation}
We can transform this integral law into a differential law governing the evolution of $\phi$. This is a well known procedure, and if you are already familiar with it, you can skip the following section.

Using the transport theorem,
\begin{equation}
	\tot{}{t} \int_\mathcal{V} \phi(r) dr = \int_\mathcal{V} \p{}{t} \phi(r) dr + \int_{\partial\mathcal{V}} \phi(r) \mathbf{v}^s(r) \cdot d \mathbf{S}
\end{equation}
where $\partial \mathcal{V}$ is the surface bounding the volume $\mathcal{V}$, $d\mathcal{S}$ an infinitesimal element of such a surface, and $\mathbf{v}^s$ the velocity of the small particles.
Using Stoke's theorem,
\begin{equation}
	\int_{\partial\mathcal{V}} \phi(r) \mathbf{v}^s(r) \cdot d \mathbf{S} = \int_\mathcal{V} \nabla \cdot \left( \mathbf{v}^s \phi(r,t) \right) d\tau 
\end{equation}

Since this is true for any volume $\mathcal{V}$ we can "remove" the integration over $\mathcal{V}$, and we have
\begin{equation}
	\p{}{t}\phi + \nabla \cdot ( \mathbf{v}^s \phi ) = 0
\end{equation}

Now we must insist on a tricky point. What is $\vs$? We said it was the velocity of the small particles. More precisely, it is the \textit{average} velocity of the small particles enclosed in a small volume $d\tau$ called a control volume\footnote{A control volume is by definition a small volume, sufficiently large for quantities such as the concentration $\phi$ to be well defined.}. We can also define a velocity $\mathbf{v}$, which is defined as the averaged velocity of all particles enclosed on the control volume $d\tau$.
What is the relation between $\mathbf{v}$ and $\vs$? If all particles were small particles, of course we would have no segregation, and thus
\begin{equation}
	\mathbf{v} = \vs
\end{equation}
If the segregation effect is not strong, we can treat it as a linear perturbation of the case without segregation:
\begin{equation}
	\vs = \mathbf{v} + \delta \textbf{v}^s
\end{equation}
We know that segregation occur only in the vertical direction, ie
\begin{align}
	\delta v^s_x = 0 \\
	\delta v^s_y = 0 
\end{align}
Moreover, we know that this small term $\delta \textbf{v}^s$ must be $0$ when all particles are small ones, but also when all particles are large ones:
\begin{align}
	\delta \textbf{v}^s(\phi = 0) = 0 \\
	\delta \textbf{v}^s(\phi = 1) = 0
\end{align}
So the simplest form for this term would be
\begin{equation}
	\delta \textbf{v}^s(\phi) = 
	\begin{pmatrix}
	0 \\
	0 \\
- q \phi( 1 - \phi ) \\
\end{pmatrix}
\end{equation}
Where $q$ is the mean segregation rate, measured experimentally.
We will use this expression, from now on\footnote{For a more rigorous - but longer! - derivation, see Appendix ?.}.

We can now rewrite the conservation law in terms of $\mathbf{v}$ and $\delta \mathbf{v}^s$:
\begin{equation}
	\p{}{t} \phi + \p{}{x} u \phi + \p{}{y} v \phi + \p{}{z} w \phi - q \p{}{z} \phi( 1 - \phi) = 0
\end{equation}
This is the equation we will use to model the segregation effect. We will call it the \textit{segregation equation}.
 
\bibliography{bib/bib.bib}{}
\bibliographystyle{plain}
\end{document}