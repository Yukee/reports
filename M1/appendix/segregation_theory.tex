\documentclass[11pt]{book}
\usepackage[left=2cm,right=2cm,top=2cm,bottom=2cm]{geometry}
%Gummi|061|=)
\title{\textbf{Segregation theory}}
\author{Nicolas}
\date{}
\begin{document}

\chapter{Segregation theory}

In this appendix, I will give a simple derivation of the equation we used to model the segregation process taking place in granular flows.

We consider a granular flow going down a slope (FIG). For the sake of simplicity, I will assume that the granular material is a mixture of small and large particles. This assumption is commonly made, since it simplifies greatly the equations, but still captures the essential physical features of segregation. For a more general theory, see CIT.
As usual, we model the granular flow using the tools of continuum mechanics. 

The idea is to see the set of large particles in the material as a matrix through the holes of which small particles can fall. 
If the material is a still, initially homogeneous mixture of small and large particles, it will partially segregate (FIG: leaving the uppermost layers full of large particles, the lowermost full of small particles.). In this case middle layers are a mixture of small and large particles.
If the material is a flowing, initially homogeneous mixture of small and large particles, it will completely segregate (FIG). This is because as the material flows down the slope, the local void ratio of the large particles matrix fluctuate, causing small to fall into newly opening gaps. The result is a progressive migration of small particles to the bottom. Because of force imbalance, large particles are pushed upward as small particles go downward. In the end, the material is constituted of a layer of pure large particles topping a layer of pure small particles.
\end{document}