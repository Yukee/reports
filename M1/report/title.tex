
\begin{titlepage}

\selectlanguage{english}
\thispagestyle{empty}



\begin{center}
\vspace{1.3cm}
{M1 de Physique
Fondamentale et Magistère 2}\\
\vspace{0.8cm}
{\Large{Internship from the 22nd of April to the 22nd of July, 2013}\\at the Manchester Centre for Nonlinear Dynamics, the University of Manchester.}\\ 
\end{center}
\vspace{1.5cm}
\begin{center}
\rule{\textwidth}{1mm}
\Huge{\Huge \textbf{Particle size segregation in granular free-surface flows}} 
\rule[0.3cm]{\textwidth}{1mm}\\
\end{center}

\vspace{0.7cm}
\begin{tabular}{lll}
~\\
~\\
~\\
~\\
~\\
~\\
~\\
~\\
~\\
~\\
~\\
~\\
~\\
~\\
~\\
~\\
~\\
~\\
~\\
~\\
~\\
~\\
~\\
~\\
~\\
~\\
~\\
~\\
~
\end{tabular}
\begin{center}
\Large{\textbf{Intern:} Nicolas Macé}\\
\Large{\textbf{Internship supervisor:} Nico Gray}\\
\end{center}

\end{titlepage}
\pagebreak 
\selectlanguage{english}
\begin{abstract}
In granular flows, large particles tend to go at the flow surface, while small particles tend to go at the flow base.
This particle size segregation is still poorly understood, especially in natural debris flows, where this effect is combined with strong sheer rates to cause lateral levees formation and flow channelisation. 

We seek to model and understand the properties of such granular flows.
We base our analysis on large-scale granular chute experiments conducted in 2011 \cite{main}.
We model the two-dimensional flow in the centre plane of the avalanche, and we propose and explanation for the observed structure. 
We develop a code to model the whole three-dimensional flow.

\selectlanguage{french}
\begin{center}
\textbf{Résumé}
\end{center}
Dans les écoulements granulaires, les grains les plus gros ont tendance à monter vers la surface, tandis que les grains les plus petits ont tendance à couler vers la base.
Cette ségrégation en milieu granulaire est encore mal comprise, particulièrement dans les laves torrentielles, dans lesquelles la combinaison de la ségrégation et du taux de cisaillement important de l'écoulement donne lieu à la formation de bourrelets latéraux canalisant ce dernier.

Nous cherchons à modéliser et à comprendre les propriétés de ces écoulements granulaires.
Nous basons notre analyse sur des expériences de grande taille conduites en 2011 \cite{main}. 
Nous modélisons l'écoulement bidimensionnel du plan centre de l'avalanche, et nous avançons une explication de la structure observée.
Nous développons un code modélisant l'écoulement tridimensionnel entier.
\end{abstract}

\selectlanguage{english}

\pagebreak

\begin{center}
\Huge \textbf{Particle size segregation in granular free-surface flows}
\end{center}


\section*{\Huge{Introduction}}

The Manchester Centre for Nonlinear Dynamics of the University of Manchester is group of researchers studying nonlinear phenomena.
The research notably includes turbulence and bifurcation in fluid dynamics, shock formation and segregation in granular materials.
Approximately 25 researchers, with physics or applied mathematics backgrounds are working inside the group. 

In nonlinear dynamics, very few theoretical results are available, and the interplay between experiments, theories and  simulations is of paramount importance. 
For example, granular media researchers, though primarily focused on theoretical modelling and computation, are also actively developing three experiments housed in the neighbouring physics building.

During my internship I studied a very specific topic: segregation in granular flows. 
But since I had no knowledge on granular materials at all when I arrived, I first took interest in understanding what is a granular material, and how we can model and understand it. \\

The first chapter of the internship report is thus a very general introduction to granular materials and granular flows. At the end of this chapter, we give qualitative ideas about the segregation effect, ideas which are developed more rigorously in the appendix.

The second chapter will focus more specifically on the experiment we seek to model and understand, and on the approach we took to model it.

The third chapter give our results and discuss them.

