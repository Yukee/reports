\documentclass[11pt]{article}
\usepackage[left=2cm,right=2cm,top=2cm,bottom=2cm]{geometry}
%Gummi|061|=)
\title{\textbf{Partial densities}}
\author{Nicolas}
\date{}
\begin{document}

\maketitle

One \footnote{Gray and Thornton, 2005, A theory for particle size segregation in shallow granular free-surface flows} of the papers you gave me to read at the beginning of my internship presents a derivation of the segregation equation I used extensively during my internship. 
So I think about including this derivation in my internship report. 
Therefore I read through the paper. The derivation is very easy to understand and to explain (no use of complicated statistical mechanics here!), but there's something I can't figure out: what exactly is a \textit{partial density}?

Let me sum up what I understood. We work with a mixture of two constituents, let's say small and large particles, labelled with $s$ and $l$ superscripts.
We have a mass and a momentum balance for each one of the two constituents. 
In this balances we will have \textit{partial} quantities: partial velocities, partial pressures, partial densities. 
I imagine the partial velocity is the velocity of the constituent. I also guess that partial pressure, let's say for the smalls $s$, is the pressure we would measure if the larges weren't there. 
But I can't think of a similar definition for the partial density. 
At first I thought that it was defined as (still for smalls)
\[
\rho^s = n^s m^s
\]
were $n^s$ is the number of small particles per unit of volume. That is, if we take a small volume $d\tau$ in the neighbourhood of $x$, and if $dN^s$ is the number of small particles in this volume,
\[
n^s(x) = \frac{dN^s}{d\tau}
\]

This definition seemed reasonable to me because then we have
\[
\rho^s + \rho^l = \frac{ dN^s m^s + dN^l m^l }{d\tau} 
= \frac{dN}{d\tau}\bar{m} = \rho
\]

The trouble is, in the paper they use
\[
\rho^s = \phi^s \rho
\]
where $\phi^s$ is the volume fraction occupied by smalls, that is $\phi^s = dn^s/(dn^s + dn^l)$.
Using my guess for $\rho^s$, we have
\[
\rho^s = n^s m^s = \frac{dn^s}{dn^s + dn^l} \frac{dn^s + dn^l}{d\tau} m^s = \phi^s \frac{dn^s + dn^l}{d\tau} m^s
\]
but $m^s (dn^s + dn^l)/d\tau = \rho$ only if $m^s = m^l$, which is obviously not true!
So, do you have an idea of the right definition of partial densities? Or do you spot any mistake in my reasoning?

Sadly there's no clue in the article. However 2 articles are cited in connection with partial quantities and mixture theory\footnote{Truesdell 1984 and Morland 1992, at the end of page 1449.}, but I can't read them, since I no longer have access to the uni's network :( 

\section{Corrections}

Okay, so I was very wrong before. I can't believe I said that
\[
\phi^s = \frac{dn^s}{dn^s + dn^l}
\]
It is of course only true if the 2 constituents of the mixture have \textit{the same volume}.  But in that case, it's not really a mixture anymore!
Otherwise,
\[
\phi^s = \frac{d\tau^s}{d\tau^s + d\tau^l} = \frac{d\tau^s}{d\tau}
\]
where $d\tau^\nu$ is the volume of $d\tau$ occupied by constituent $\nu$. 

And now, if we call $\rho^{*\nu}$ the intrinsic density of constituent $\nu$, and if we neglect the effect of the interstitial fluid (or if we incorporate it into one of the 2 phases),
\[
\rho = \frac{d\tau^s \rho^{*s} + d\tau^l \rho^{*l}}{d\tau} = \rho^s + \rho^l
\]
thus we have 
\[
\rho^\nu = \frac{d\tau^\nu}{d\tau} \rho^{*nu} = \phi^\nu \rho^{*\nu}
\]
as in the paper. Yeah! It finally makes sense to me...
\end{document}
